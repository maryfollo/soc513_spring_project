%%%%%%%%%%%%%%%%%%%%%%%%%%%%%%%%%
% This is a slightly modified template of the one built by
% Steven V. Miller. Information can be found here:
%  http://svmiller.com/blog/2016/02/svm-r-markdown-manuscript/
%
% I added the use of raggedright to the anonymous option
% because journals,  the ability to put all the footnotes
% in endnotes, and the ability to manually adjust
% the starting page from the YAML header.
%
% Here are the options that you can define in the YAML
% header.
%
% fontfamily - self-explanatory
% fontsize - self-explanatory (e.g. 10pt, 11pt)
% anonymous - true/false. If true, names will be supressed and the
%                       text will be double-spaced and ragged
%                       right. For submission. 
% endnotes - true/false. If true, the footnotes will be put in a
%                   section at the end just ahead of the references.  
% keywords - self-explanatory
% thanks - shows up as a footnote to the title on page 1
% abstract - self explanatory
% appendix - if true, tables and figures will have  in
%                   front
% appendixletter - The letter to append to tables and figures in
%                             appendix
% pagenumber - Put in a number here to get a starting page number
%                         besides 1. 
%%%%%%%%%%%%%%%%%%%%%%%%%%%%%%%%%%


\documentclass[11pt,]{article}
\usepackage[left=1in,top=1in,right=1in,bottom=1in]{geometry}
\usepackage{amsmath}
\usepackage{float}
\usepackage{dcolumn}

\newcommand*{\authorfont}{\fontfamily{phv}\selectfont}
\usepackage[]{mathpazo}


  \usepackage[T1]{fontenc}
  \usepackage[utf8]{inputenc}



\usepackage{abstract}
\renewcommand{\abstractname}{}    % clear the title
\renewcommand{\absnamepos}{empty} % originally center

\providecommand{\tightlist}{%
  \setlength{\itemsep}{0pt}\setlength{\parskip}{0pt}}

\renewenvironment{abstract}
 {{%
    \setlength{\leftmargin}{0mm}
    \setlength{\rightmargin}{\leftmargin}%
  }%
  \relax}
 {\endlist}

\makeatletter
\def\@maketitle{%
  \newpage
%  \null
%  \vskip 2em%
%  \begin{center}%
  \let \footnote \thanks
    {\fontsize{18}{20}\selectfont\raggedright  \setlength{\parindent}{0pt} \@title \par}%
}
%\fi
\makeatother




\setcounter{secnumdepth}{0}



\title{Does the Media Influence Voters: A Study of the 2012 and 2016 Elections \thanks{Thank you to the University of Oregon, Political Science department for
forcing me to take this class and the Soc 513 family for showing up
every Tuesday and Thursday with sub-par enthusiasm. Also a big thanks to
the UO Econ department who showed me a million different ways to work
with R, all which confused me more. Shoutout to Chris from Econ for
being GitHub friends with me. And a final huge thank you to Aaron who
put up with so much, you're the real MVP.}  }



\author{\Large Mary Follo\vspace{0.05in} \newline\normalsize\emph{University of Oregon, Political Science}   \and \Large Aaron Gullickson\vspace{0.05in} \newline\normalsize\emph{University of Oregon, Sociology}  }


\date{}

\usepackage{titlesec}

\titleformat*{\section}{\normalsize\bfseries}
\titleformat*{\subsection}{\normalsize\itshape}
\titleformat*{\subsubsection}{\normalsize\itshape}
\titleformat*{\paragraph}{\normalsize\itshape}
\titleformat*{\subparagraph}{\normalsize\itshape}


\usepackage{natbib}
\bibliographystyle{./resources/ajs.bst}


%\renewcommand{\refname}{References}
%\makeatletter
%\renewcommand\bibsection{
%    \section*{{\normalsize{\refname}}}%
%}%
%\makeatother

\newtheorem{hypothesis}{Hypothesis}
\usepackage{setspace}

\makeatletter
\@ifpackageloaded{hyperref}{}{%
\ifxetex
  \usepackage[setpagesize=false, % page size defined by xetex
              unicode=false, % unicode breaks when used with xetex
              xetex]{hyperref}
\else
  \usepackage[unicode=true]{hyperref}
\fi
}
\@ifpackageloaded{color}{
    \PassOptionsToPackage{usenames,dvipsnames}{color}
}{%
    \usepackage[usenames,dvipsnames]{color}
}
\makeatother
\hypersetup{breaklinks=true,
            bookmarks=true,
            pdfauthor={Mary Follo (University of Oregon, Political Science) and Aaron Gullickson (University of Oregon, Sociology)},
             pdfkeywords = {Media Influence, Voting, Elections},  
            pdftitle={Does the Media Influence Voters: A Study of the 2012 and 2016 Elections},
            colorlinks=true,
            citecolor=blue,
            urlcolor=blue,
            linkcolor=magenta,
            pdfborder={0 0 0}}
\urlstyle{same}  % don't use monospace font for urls

\usepackage{endnotes}


\newlength{\normalparindent}
\setlength{\normalparindent}{\parindent}

%prettier captions for figures and tables
%I am making the text of figure captions smaller but not table captions
\usepackage[labelfont=bf,labelsep=period]{caption}
\captionsetup[figure]{font=footnotesize}

\begin{document}
	
% \pagenumbering{arabic}% resets `page` counter to 1 
%
\setcounter{page}{1}

% \maketitle

{% \usefont{T1}{pnc}{m}{n}
\setlength{\parindent}{0pt}
\thispagestyle{plain}
{\fontsize{18}{20}\selectfont\raggedright 
\maketitle  % title \par  

}

{
   \vskip 13.5pt\relax \normalsize\fontsize{11}{12} 
\textbf{\authorfont Mary Follo} \hskip 15pt \emph{\small University of Oregon, Political Science}   \par \textbf{\authorfont Aaron Gullickson} \hskip 15pt \emph{\small University of Oregon, Sociology}   

}

}







\begin{abstract}

    \hbox{\vrule height .2pt width 39.14pc}

    \vskip 8.5pt % \small 

\noindent As the Unites States has become more polarized with intense and
contested elections, understanding the impact that variables like media
have on voting behavior are becoming more and more important. This study
uses ANES data from the 2012 and 2016 election to compare the effect
that attention to news media as well as age and party affiliation has on
voting. I found small effects that lack of media attention does result
in higher rates of voting for a republican candidate. Additionally, data
shows that younger Democrats are more likely to follow news media than
older Democrats and older Republicans are more likely to pay attention
than younger Republicans. Further I was able to find that media
attention has increased between the two elections, proving that greater
research is needed to determine how elections can be influenced by
exogenous variables like media.


\vskip 8.5pt \noindent \emph{Keywords}: Media Influence, Voting, Elections \par

    \hbox{\vrule height .2pt width 39.14pc}



\end{abstract}


\vskip 6.5pt

\noindent  \section{Introduction}\label{introduction}

The role of media in voting has always played a significant role in U.S.
politics. From political propaganda pamphlets during the revolutionary
war to the political machine era; parties, candidates and their
campaigns have worked to exert influence over voters. Today, the media
is one of the strongest resources candidates have at their disposal to
mobilize their voters. All aspects of campaigns are broadcast on
television, the Internet and social media in an attempt to influence
behaviors. In fact, recent research, when studying the effect of Fox
news bias, has shown that the media can have a sizable political impact
(DellaVigna and Kaplan, 2006). In the 2008 election for example, Oprah
Winfrey's celebrity endorsement of Barack Obama on her television show
can be correlated with an extra one million votes during the election.
Hence, the ``Oprah Effect'' or ability for democracy to be strengthened
among low-awareness voters through the consumption of soft news media
was created (Baum, 2006). At the time of the 2008 election however, a
new media type was just taking shape. Social media that is now heavily
utilized for campaigns was just becoming accessible to users with
Facebook and Twitter taking shape in 2006 and Instagram in 2008. These
new forms of social media are just one more way that candidates and
campaigns can exert influence over voters.

Understanding the role of media in public opinion and behavior has
historically been hotly debated among three distinct models: the
hypodermic model, the minimum effects model and the constructivist
model. The hypodermic model asserts that ``the mass media exercised a
powerful and persuasive influence'' on a public that was ``inherently
susceptible to manipulation'' (Curran, Gurevitch and Woollacott, 1982).
Post-World War I, scholars theorized that the media was seen as a
brainwashing agent that preyed on the unknowing public mind that could
be completely consumed by media agency. This pessimistic theory has
proven empty over time and does not take into account individual agency
but instead assumes each person is a blank slate, a tabula rasa that is
waiting for the media or some other influence to give them opinions. In
the 1950's and 60's a new school of thought emerged called the minimal
effects theory. Introduced by Joseph Klapper, the minimal effects theory
finds that ``only a tiny fraction of voters actually changed their vote
intentions during an election campaign, that audience motivations and
prior beliefs influenced the interpretations of persuasive messages''
(Neuman\& Guggenheim, 2011). The idea of media manipulation shifted from
``what the media does to people to what the people do with the media''
(Gamson, 1988). In this way, the minimal effects approach tells a
narrative about a voter with more agency but does not take into account
the real role that media framing and priming play in political behavior.
These two early theories of media influence portrayed the American
citizen as an all of nothing individual, either completely consumed by
the media or completely obstinate to it.

Instead, I focus on the use of the constructivist model that ``takes
into account the changing nature of the media's power as it interacts
among voters who influence each other while contributing to the
construction of messages, meanings and outcomes in the electoral
process'' (Gamson, 1988). The constructivist model allows one's personal
social discourse to align with the media discourse that one is
influenced by to construct their own opinions and vote choice by melding
information obtained from the media with their existing beliefs (Neuman
et. al, 1992). In this way, citizens are able to consume a large variety
of news media and construct their own political knowledge out of what
they believe is most relevant. The agenda setting and priming that is
absent in the minimal effects approach is welcomed in the constructivist
approach and shows that the media can be both directly and indirectly
persuasive through the frames it chooses to shape discourse with
(Armoudian \& Crigler, 2010). As technology continues to develop, more
factors will be added to the complex interplay that constructs each
individual's political discourse. Political meaning and the construction
of personal voting behavior then is not solely determined by media
effect, nor does it have a minimal effect. Media, especially social
media can be effective helping shape one's thoughts through its news
generating ability, however when it comes to actual decision making and
voter behavior, individuals tend to rely on their interpersonal networks
for direction.

As mass media like social media have gained prominence in American
society, citizens have become dependent upon them for a variety of news
information. The reliance on social media especially by youths for news
relates to Ball-Rokeach and DeFleur's dependency theory. Dependency
theory states that, ``audiences are dependent on media information
resources which leads to modifications in both personal and social
processes'' (Ball-Rokeach and DeFleur, 1976). What I believe to be an
extension of the constructivist model, dependency theory not only states
media's role in helping create discourses but it gives power to mass
media by acknowledging the increasing reliance upon it for information
and resources. ``In American society, the media are presumed to have
several unique functions. They operate as a Fourth estate gathering and
delivering information about the actions of the government;
{[}\ldots{}{]} they constitute the principle source of the ordinary
citizen's conceptions of national and world events; they provide
enormous amounts of entertainment'' (B-R and DeFleur, 1976). Gaining a
wide variety of important information however does not mean that people
are influenced, ``Topics are filtered through media
information-gathering and --processing systems and then selectively
disseminated. The public then sorts out their interest and concern with
this information as a function of both their individual differences in
personal make-up and their location in societal strata and categories''
(B-R and DeFleur, 1976).

Despite what has been characterized in the literature by many American
behavioralists, interpersonal effects influence a person's
decision-making and attitude more than the information they receive from
the media. ``American voters do not operate in the social vacuum that
much of the contemporary voting literature seems to assume. Rather,
voters' enduring personal characteristics interact with messages they
are receiving from the social context in which they operate
{[}\ldots{}{]} interpersonal discussion outweighs the media in affecting
the vote'' (Beck et. al, 2002). Additionally, recent literature has
focused more on the effect that position in one's social network has a
greater effect on voting behavior than does media influence. ``The
dependencies people have on media information are a product of the
nature of the socio-cultural system, category membership, individual
needs, and then number and centrality of the unique information
functions that the media system serves for individual and society
{[}\ldots{}{]} when people's social realities are entirely adequate
before and during message reception, media messages may have little to
no alteration effects'' (B-R, DeFleur, 1976).

Operating under a constructivist model and dependency theory one can
begin to understand how media is just one factor of a greater social
system that can shape one's attitude and behavior. The 2012 and 2016
elections are two great case studies for this project because of the
prominence of both traditional forms of media and social media. Given
the attention paid to the 2016 election in regards to ``fake news'',
media bombshells and negative ads, we should expect media attention to
have increased from the 2012 election. Within the scope of this paper, I
believe that while media attentiveness has increased from 2012 to 2016,
I do not believe this impacts one type of voter or one party more than
the other. Further I believe that the accessibility of media to younger
voters will result in more media attention paid by younger voters and
expect media attention to taper off with older voters.

\section{Data and Methods}\label{data-and-methods}

For this project I used ANES Time Series data from both the 2012 and
2016 election surveys. The 2012 Time Series study data was collected
between September 2012 and January 2013, with respondents being
interviewed two months prior to the 2012 election as well as the day
after the election. A voter validation data set was added and will be
merged with the Time Series data to to validate self-reported voter
turnout and registration from the Time Series data. This data includes
content on electoral participation, voting behavior and public opinion.

The 2016 Time Series study is similar to the 2012 but has additional
features including supplemental data pulled from respondents Facebook
accounts. Additionally, the Methodology file includes data on
households, interviews and records of contact among respondents. The
Address data gives voter turnout status for adults at each address that
was sampled in the Time Series study.

I created a subset of the following variables in the ANES 2016 dataset:
vote choice, how often they use media news to follow politics, gender,
age, state, party affiliation, how many days do they use social media
and how closely they follow politics in the media. I created a subset of
the following variavles in the ANES 2012 dataset: vote choice, state,
gender, age, how often they use TV news to follow politics, how often
they use the Internet to follow politics, party affiliation and how
often they are interested in politics.

After created subsets of each data set I recoded all the variables, in
some cases I combined answers and omitted non-responses including
turning age into a continuous variable. For my models I was able to
subset the data again to include only voting ages and worked on
combining rows and columns in certain variables to merge matrices. The
graphs created include bivariate analysis of candidate choice across age
groups as well as media attention across age groups and media attention
by party affiliation. Finally, I used an interaction term to create a
multinomial model to determine if voting by different levels of media
attention was significant.

\section{Results}\label{results}

\begin{figure}

{\centering \includegraphics{main-Intro__files/figure-latex/vote_age_2012-1} 

}

\caption{Graph 1: Proportion of Candidate Choice across Age Group Means 2012}\label{fig:vote_age_2012}
\end{figure}

\begin{figure}

{\centering \includegraphics{main-Intro__files/figure-latex/vote_age_2016-1} 

}

\caption{Graph 2: Proportion of Candidate Choice across Age Group Means}\label{fig:vote_age_2016}
\end{figure}

The first two graphs show a bivariate analysis of the proportion of
candidate choice across age group means. For this analysis, each age
group was subset to include a range of five years. The 2012 graph shows
that for all age groups, except age 72, Obama was the overwhelming
favorite. The youngest group of voters chose to vote for Obama at much
higher proportions (40\% compared to 10\%). While the proportion of
votes for Romney begins to increase when a person enters their 40's,
Obama is there clear winner in almost all age groups. The 72-year-old
age group does show a slightly higher probability of voting for Romney
compared to Obama, but for the 80 year old age group; there are equal
proportions of votes for each candidate. In the 2016 data, proportions
are closer between the two candidates with Clinton capturing around 25\%
of votes from the youngest age group while Trump received around 15\%.
The proportion of votes for Trump increases at an early age than did
votes for Romney in 2012. In the age range of 47, one becomes slightly
more likely to vote for Trump. While proportions fluctuate, Trump
maintains the advantage for most age groups over 42. This of course is
no surprise for most scholars of American voting behavior. The
Republican Party overwhelmingly targets an older demographic with more
conservation policies than the Democratic party. What is interesting
however is that the proportion difference is much closer in 2016 than it
was in 2012, this can be for a variety of reasons including incumbent
advantage or the progressive pull of Obama's neo-liberal policies on
Republicans. What it does appear to confirm however is that the older
one gets, the more likely one is to vote Republican.

\begin{figure}

{\centering \includegraphics{main-Intro__files/figure-latex/media_age_2012-1} 

}

\caption{Graph 3: Proportion of Media Attention for Internet Users Across Age Group Means 2012}\label{fig:media_age_2012}
\end{figure}

\begin{figure}

{\centering \includegraphics{main-Intro__files/figure-latex/media_age_2016-1} 

}

\caption{Graph 4: Proportion of Media attention across Age Group Means 2016}\label{fig:media_age_2016}
\end{figure}

The next set of graphs explores the relationship between age and the
amount of attention one pays to politics in the media. In 2012, for
voters between the ages of 19 and 35, most paid little to no attention
to politics in the media. As age increases, the amount of voters paying
little to no attention decreases, as do most of the categories. This
makes sense, as older people are less likely to use media like the
Internet to consume information about politics. Further, in 2012, media
was less accessible to older populations than it is now. In 2016,
proportions for media attention are much higher overall. Still, little
to none, and a moderate amount are the categories that best characterize
young people and politics in the media. However, starting in one's late
30's paying attention to politics in the media increases and the 72 year
old age group reporting the highest proportion of users paying ``A lot''
of attention. While this is a somewhat unusual result, I believe this
can be attributed to media becoming the primary way people receive news.
Out are of the days of newspaper, radio and even television news, as
more and more people consumer Internet news. As media has become more
accessible to older generations, it should not be surprising that older
people use media to become informed about politics than younger
generations who are more interested in popular apps and websites.

\begin{figure}

{\centering \includegraphics{main-Intro__files/figure-latex/media_party_2012-1} 

}

\caption{Graph 5: Proportion of Media Attention by Internet Users across Parties 2012}\label{fig:media_party_2012}
\end{figure}

\begin{figure}

{\centering \includegraphics{main-Intro__files/figure-latex/media_party_2016-1} 

}

\caption{Graph 6: Proportion of Media Attention across Parties 2016}\label{fig:media_party_2016}
\end{figure}

My final bivariate analysis plots look at media attention by party
affiliation. As shown above in Graphs 1and 2, voting Republican seems to
increase over age and attention to politics using media also seems to
increase by age. However, I also want to look at party affiliation and
media interest. Do we see these same results by party and media
attention, or are age, voting and media not as contingent on each other?
Further, voting and media attention both have higher proportions in 2016
than in 2012, by exploring more variables, we can understand how much or
how little of an impact the growing media has on voting. In 2012,
Republicans paid greater attention to the media than did Democrats
except in the categories of ``A little'' and ``None''. This is
interesting that Republicans would pay more attention gives that Obama
won out in almost all ages. Furthermore, in 2016, Republicans have
slightly higher rates of paying a great deal of attention, but Democrats
pay more attention in the ``A lot'' and ``A moderate amount''
categories. Rates in 2016 are also much larger than rates in 2012 with
each response gaining nearly 30\% of responses compared to the 20\% and
under than answers received in 2012. What this could show is that the
opposition party pays more attention to the news media than does the
incumbent party. Further, in 2016, Democrats who were trying to mobilize
a broad base, paid more attention than did homogenous Republicans.

\begin{verbatim}
## # weights:  40 (27 variable)
## initial  value 5538.245973 
## iter  10 value 4552.050916
## iter  20 value 4424.653589
## iter  30 value 4249.031932
## iter  40 value 4237.636477
## final  value 4237.636395 
## converged
\end{verbatim}

\begin{verbatim}
##        (Intercept) media_intINT1 media_intINT2 media_intINT3 I(age - 40)
## Obama       0.4693        0.6989        0.4937        0.4289      0.0336
## Other      -2.8129        0.1924        0.7899        0.4520      0.0196
## Romney     -0.1255        0.6264        0.5801        0.2908      0.0444
##        genderMale media_intINT1:I(age - 40) media_intINT2:I(age - 40)
## Obama     -0.2426                   -0.0043                   -0.0079
## Other      0.8543                    0.0051                   -0.0626
## Romney    -0.0558                   -0.0019                   -0.0032
##        media_intINT3:I(age - 40)
## Obama                     0.0156
## Other                     0.0334
## Romney                    0.0228
\end{verbatim}

\begin{figure}

{\centering \includegraphics{main-Intro__files/figure-latex/model_2012-1} 

}

\caption{Graph 7: Proportion of Media Attention by Candidate Choice across Age}\label{fig:model_20121}
\end{figure}\begin{figure}

{\centering \includegraphics{main-Intro__files/figure-latex/model_2012-2} 

}

\caption{Graph 7: Proportion of Media Attention by Candidate Choice across Age}\label{fig:model_20122}
\end{figure}

\begin{verbatim}
## # weights:  40 (27 variable)
## initial  value 3596.047573 
## iter  10 value 2801.293013
## iter  20 value 2571.605316
## iter  30 value 2440.786040
## iter  40 value 2437.103943
## final  value 2437.101944 
## converged
\end{verbatim}

\begin{verbatim}
##             (Intercept) media_att1 media_att2 media_att3 I(age - 40)
## No vote        -3.50844   -0.22924    0.21725    0.13669     0.01122
## Third party    -1.27872   -0.25480   -0.22354   -0.34588    -0.01126
## Trump           0.05548   -0.39018   -0.30564    0.25748     0.01043
##             genderMale media_att1:I(age - 40) media_att2:I(age - 40)
## No vote        0.17024               -0.01154               -0.00762
## Third party   -0.14179                0.00551               -0.01874
## Trump          0.16273                0.00800                0.00770
##             media_att3:I(age - 40)
## No vote                   -0.01186
## Third party               -0.01564
## Trump                     -0.01461
\end{verbatim}

\begin{figure}

{\centering \includegraphics{main-Intro__files/figure-latex/model-1} 

}

\caption{Graph 8: Proportion of Media Attention by Candidate Choice across Age}\label{fig:model1}
\end{figure}\begin{figure}

{\centering \includegraphics{main-Intro__files/figure-latex/model-2} 

}

\caption{Graph 8: Proportion of Media Attention by Candidate Choice across Age}\label{fig:model2}
\end{figure}

Graphs 7 and 8 show the proportion of media attention by candidate
choice across age with the interaction effect of age and controlling for
gender. With the interaction effect, we get a slightly different graph
than shown above. With younger voters having a high proportion of media
attention when voting for Obama. The highest media attention and the
lowest converge at around 70 years old and begin to drop off. For Romney
voters however, young people have lower proportions with the highest and
lowest media attention both around 30 years of age. As the age of Romney
voters increases, so does attention to news media. Further, those who
pay a great deal of attention and those who pay no attention are not as
divergent as the rates for Obama supporters. This shows that in general
Romney voters pay less attention to news media than do Obama voters. In
2016, Clinton supporters like the Obama supporters in 2012, are more
engaged in news media at younger ages. In fact, news media attention
decreases where it begins to converge around the age of 80; much older
than that of Obama supporters. Trump, like Romney, has the opposite
effect with the highest rate of media attention being ``A little to
none'' at nearly 40\% for young voters. While a little to none maintains
being one of the highest rates across age, all media attention increases
over age, converging around 70 years old. What this shows is that for
Trump supporters is that being uninformed characterizes many young
voters and again that voting for a Republican increases as one ages.
Unlike Romney voters whose rates of any media attention were relatively
low for young voters, Trump supports paid more attention to news media
at all ages. This could be indicative of increasing availability of
media as well as the amount of media sources that now cover politics.

\begin{verbatim}
## # weights:  36 (24 variable)
## initial  value 2082.214130 
## iter  10 value 1979.051417
## iter  20 value 1953.600402
## iter  30 value 1953.448357
## iter  30 value 1953.448348
## iter  30 value 1953.448348
## final  value 1953.448348 
## converged
\end{verbatim}

\begin{verbatim}
##                   (Intercept) partyregNone or Independent partyregOther
## A moderate amount  0.28807958                 -0.06225995    -0.7514523
## A lot             -0.01812868                 -0.39844910    -0.4569703
## A great deal      -1.62741552                 -0.09598497    -0.7324503
##                   partyregRepublican Party voteNo vote voteThird party
## A moderate amount                0.3137500  -0.7065591     -0.09606184
## A lot                            0.1646701  -0.1827446     -0.49653460
## A great deal                     0.3132355  -0.7572653     -0.97928091
##                    voteTrump        age
## A moderate amount -0.4451293 0.01314081
## A lot             -0.3084424 0.02028215
## A great deal      -0.4880246 0.04890139
\end{verbatim}

\begin{verbatim}
## 
## =========================================================================
##                             A moderate amount  A lot         A great deal
## -------------------------------------------------------------------------
## A great deal                    0.29              -0.02         -1.63 ***
##                                (0.28)             (0.28)        (0.31)   
## A lot                          -0.06              -0.40         -0.10    
##                                (0.23)             (0.23)        (0.24)   
## A moderate amount              -0.75              -0.46         -0.73    
##                                (0.73)             (0.70)        (0.82)   
## Interaction (age-40)            0.31               0.16          0.31    
##                                (0.26)             (0.26)        (0.27)   
## Male                           -0.71              -0.18         -0.76    
##                                (0.62)             (0.58)        (0.67)   
## A great deal (age-40)          -0.10              -0.50         -0.98 *  
##                                (0.31)             (0.34)        (0.39)   
## A lot (age-40)                 -0.45              -0.31         -0.49 *  
##                                (0.23)             (0.24)        (0.24)   
## A moderate amount (age-40)      0.01 *             0.02 ***      0.05 ***
##                                (0.01)             (0.01)        (0.01)   
## -------------------------------------------------------------------------
## AIC                          3954.90            3954.90       3954.90    
## BIC                          4082.45            4082.45       4082.45    
## Log Likelihood              -1953.45           -1953.45      -1953.45    
## Deviance                     3906.90            3906.90       3906.90    
## Num. obs.                    1502               1502          1502       
## =========================================================================
## *** p < 0.001, ** p < 0.01, * p < 0.05
\end{verbatim}

Finally looking only at the 2016 election, Table 1 shows a regression of
media attention by party registration holding vote choice and age
constant. In this table, voting for Clinton and being female is the
constant. What we see is that for ``A great deal'' and ``A lot'' of
media attention, as those rates go down, voting for Trump increases. Put
differently, paying less attention to news media makes one more likely
to vote for Trump. These two variables are statistically significant.
Additionally, being male and being older makes one more likely to vote
for Trump. At the bottom of Table 1, paying ``A moderate amount'' of
media attention with the interaction of age is shown to be significant,
because this result includes all of the other variables in the model, it
is likely that collinearity produced this result.

\section{Conclusions}\label{conclusions}

Hence, what the data tells us is that media effects matter based on age
and party affiliation. Young democrats pay more attention to the media
to get their political news while older democrats play less attention to
forms of media. Alternatively, young republicans pay little attention to
the media for political news while older republicans pay a great deal
more attention. Possible reasons for these results include the fact that
the Democratic Party targets younger people with more progressive
policies and is able to better mobilize influential groups and people to
help campaign. We see this mobilization of young people through
celebrity endorsement and campaign related text communication.
Republicans on the other hand promote far more conservative policies
that appeal to an older crowd. Specifically in 2016, young people not
paying attention to politics in the media helped garner Trump support in
the polls, as he was the subject of many controversies. Young people who
got their news from older relatives or like-minded friends would not
receive the same political information as those who watched the news or
even read twitter. Therefore, this data confirms the bias and
stereotypes of the different parties when gathering information to vote.
While this result is plausible, more data on voting behavior is needed
to correlate the results. Further the lack of specific behavior based
questions by the ANES could result in skewed data. Given the fact the
``not available'' or no answer was a widely answered response in the
surveys, it is possible that this impacted the results especially in the
data on proportions. With more time, a factor analysis would have been
beneficial as well as using a multiple imputation model to impute the
extensive missing values. Finally, more data is needed to understand the
impact of specific types of media on voters.

Therefore, this project confirmed my hypothesis that media impact has
increased since 2012 and of media effects, but only for the Democratic
Party. My hypothesis that media would not affect one party more than the
other is essentially disproved by this data, as results were different
between the two parties. More research is needed on the Republican Party
and how their information is gathered specifically among younger voters.
Looking back on the constructivist model of media effects, it is
possible that younger voters gather political information from
interpersonal sources as opposed to the media, which results in their
voting behavior. As campaigns and parties become more polarized and
contested and the role of media in our lives grow, understanding how
different variables like how the media impacts political behavior will
be key to American politics.

\section{References}\label{references}

Armoudian, M., Crigler, A. N., \& Leighley, J. E. (2010). Constructing
the vote: Media effects in a constructionist model. In The Oxford
handbook of American elections and political behavior. Oxford University
Press.

Ball-Rokeach, S. J., \& DeFleur, M. L. (1976). A dependency model of
mass-media effects. Communication research, 3(1), 3-21.

Baum, M. A., \& Jamison, A. S. (2006). The Oprah effect: How soft news
helps inattentive citizens vote consistently. The Journal of Politics,
68(4), 946-959.

DellaVigna, S., \& Kaplan, E. (2007). The Fox News effect: Media bias
and voting. The Quarterly Journal of Economics, 122(3), 1187-1234.

Gamson, W. A. (1988). The 1987 distinguished lecture: A constructionist
approach to mass media and public opinion. Symbolic interaction, 11(2),
161-174.

Garthwaite, C., \& Moore, T. J. (2012). Can celebrity endorsements
affect political outcomes? Evidence from the 2008 US democratic
presidential primary. The journal of law, economics, \& organization,
29(2), 355-384.

Gurevitch, M., Bennett, T., Curran, J., \& Woollacott, J. (Eds.).
(1982). Culture, society and the media (Vol. 759). London: Methuen.

Klapper, J. T. (1960). The effects of mass communication.

Neuman, W. R., \& Guggenheim, L. (2011). The evolution of media effects
theory: A six-stage model of cumulative research. Communication Theory,
21(2), 169-196.

Zaller, J. R. (1992). The nature and origins of mass opinion. Cambridge
university press



\bibliography{../project.bib}
\end{document}
